\documentclass[11pt, a4paper]{article}
%\usepackage{geometry}
\usepackage[inner=1.5cm,outer=1.5cm,top=2.5cm,bottom=2.5cm]{geometry}
\pagestyle{empty}
\usepackage{graphicx}
\usepackage{amsmath, amssymb}
\usepackage[usenames,dvipsnames]{color}
\definecolor{darkblue}{rgb}{0,0,.6}
\definecolor{darkred}{rgb}{.7,0,0}
\definecolor{darkgreen}{rgb}{0,.6,0}
\definecolor{red}{rgb}{.98,0,0}
\usepackage[colorlinks,pagebackref,pdfusetitle,urlcolor=darkblue,citecolor=darkblue,linkcolor=darkred,bookmarksnumbered,plainpages=false]{hyperref}
\renewcommand{\thefootnote}{\fnsymbol{footnote}}



\thispagestyle{plain}

%%%%%%%%%%%% LISTING %%%
\usepackage{listings}
\usepackage{caption}
\DeclareCaptionFont{white}{\color{white}}
\DeclareCaptionFormat{listing}{\colorbox{gray}{\parbox{\textwidth}{#1#2#3}}}
\captionsetup[lstlisting]{format=listing,labelfont=white,textfont=white}
\usepackage{verbatim} % used to display code
\usepackage{fancyvrb}
\usepackage{acronym}
\usepackage{amsthm}
\VerbatimFootnotes % Required, otherwise verbatim does not work in footnotes!



\definecolor{OliveGreen}{cmyk}{0.64,0,0.95,0.40}
\definecolor{CadetBlue}{cmyk}{0.62,0.57,0.23,0}
\definecolor{lightlightgray}{gray}{0.93}



\lstset{
%language=bash,                          % Code langugage
basicstyle=\ttfamily,                   % Code font, Examples: \footnotesize, \ttfamily
keywordstyle=\color{OliveGreen},        % Keywords font ('*' = uppercase)
commentstyle=\color{gray},              % Comments font
numbers=left,                           % Line nums position
numberstyle=\tiny,                      % Line-numbers fonts
stepnumber=1,                           % Step between two line-numbers
numbersep=5pt,                          % How far are line-numbers from code
backgroundcolor=\color{lightlightgray}, % Choose background color
frame=none,                             % A frame around the code
tabsize=2,                              % Default tab size
captionpos=t,                           % Caption-position = bottom
breaklines=true,                        % Automatic line breaking?
breakatwhitespace=false,                % Automatic breaks only at whitespace?
showspaces=false,                       % Dont make spaces visible
showtabs=false,                         % Dont make tabls visible
columns=flexible,                       % Column format
morekeywords={__global__, __device__},  % CUDA specific keywords
}

%%%%%%%%%%%%%%%%%%%%%%%%%%%%%%%%%%%%
\begin{document}
\begin{center}
  {\Large \textsc{Problem Set 4: Diff-in-diff}}

  MGMT 737
\end{center}


\begin{enumerate}
\item \textbf{Diff-in-diff:} Consider the following simulated data in
  \texttt{dind\_data.csv}. There are 10 time periods (\texttt{timeid}),
  1000 units (\texttt{ids}), and the treatment turns on in period 5
  (\texttt{post}). The treated group (\texttt{treated\_group}) receives
  the treatment in period 5 and the control group does not. The
  treatment is fully absorbing. For this exercise, you may use canned
  linear regression packages, or your own constructed
  estimator. Please be specific on what standard errors you are
  reporting. 
  \begin{enumerate}
  \item First, focus on \texttt{y\_instant} and estimate the treatment effect for the following three regressions:
    \begin{itemize}
    \item The treatment effect, controlling for group status and the post period $y_{it} = \alpha_{post} + \alpha_{treated} + \beta \texttt{post} \times \texttt{treated}$
    \item The treatment effect, controlling for group status and time fixed effects $y_{it} = \alpha_{t} + \alpha_{treated} + \beta \texttt{post} \times \texttt{treated}$      
    \item The treatment effect, controlling for unit and time fixed effects $y_{it} = \alpha_{t} + \alpha_{i} + \beta \texttt{post} \times \texttt{treated}$            
    \end{itemize}
    Compare the point estimates and standard errors. How do they differ? Why? Plot the means of the outcome over time, split out by treatment group, to help explain your answer.
  \item Now, we want to estimate the dynamic effects for this
    outcome. Estimate the effect of the treatment relative to time
    period 4, for all time periods, controlling for unit and time
    fixed effects. Plot these coefficients, and report the estimate
    and standard error in period 6.

    \textbf{N.B. Verify to yourself that you must omit the estimated
      effect in a given time period.  What would your estimator
      estimate if you failed to omit a given period?}
  \item How does your point estimate from part a with the same
    specification compare to taking the simple average of the dynamic
    coefficients from the regression in part b? 
  \item Now estimate the effect of the treatment relative to time
    period 3. How does the estimate change? Use your figure to explain why.
  \item Now, we want to estimate the effect for the \texttt{y\_dynamic}
    outcome. Replicate part a and b using the this outcome. Explain
    why this outcome looks different (consider plotting the means as
    before).
  \item Now replicate part b and d using \texttt{y\_dynamic2}. Do you
    think \texttt{y\_dynamic2} satisfies the necessary condiitions for
    diff-in-diff?
  \item Consider the standard errors that you chose in this
    exercise. Repeat part a, but try with both robust standard errors,
    and clustered by id. Report the difference in standard errors for
    the estimate of the treatment effect, controlling for unit and
    time fixed effects.

    To get intuition on this result, do the following exercise:

    \begin{itemize}
    \item Calculate the residuals from the specification
    \item Estimate the autocorrelation within unit by period
    \end{itemize}
    What does the autocorrelation structure imply about how the shocks
    line up with the treatment timing?
  \end{enumerate}
  
\item\textbf{Event Study} Next, we consider an event study
  approach. We will use data from Sun and Abraham (2021)'s
  application, which replicates results from Dobkin et al. (2020)'s
  results using the HRS data (which is publicly available). Variables:
  \texttt{hhidpn} household identifier -- this is the identifier for
  an individual in the panel; \texttt{wave} time identifier (wave of
  survey) -- this is the time index of the survey; \texttt{wave\_hosp}
  time of event -- time when the individual is hospitalized; and
  \texttt{oop\_spend} Out-of-pocket spending.
  \begin{enumerate}
  \item We will be following Sun and Abraham's notation for describing
    this setup. Denote the initial time period of treatment for a unit
    as $E_{i}$. What variable corresponds to $E_{i}$ in our dataset?
    Construct a variable $D_{it} = 1(E_{i} <= t)$ which is equal to
    one when an individual is treated. What share of individuals are
    treated in period 7,8,9,10?
  \item Estimate the traditional static two-way fixed effects
    estimation for this setup:
    \begin{equation}\label{eq:twfe}
      Y_{it} = \alpha_{i} + \lambda_{t} + D_{it}\beta + \epsilon_{it}
    \end{equation}
    where $Y_{it}$ is \texttt{oop\_spend}, $\alpha_{i}$ is a unit
    fixed effect and $\lambda_{t}$ is a time fixed effect. Report the
    estimate for $\beta$ and its standard error (adjust for
    appropriate inference in the panel setting). 
  \item Now, consider the estimation group by group. Denote our
    control group as the last group ever treated. For each other
    treated wave, estimate the treatment effect relative to this
    group, excluding the last period of data. Report the coefficients
    and standard errors for each of these waves. How do these
    estimates compare to your last result? For Wave 8 cohort, what is the
    relative comparison period for the diff-in-diff? In other words,
    the diff across units is Cohort Wave 8 vs. Cohort Wave 11. What is the diff
    across time comparing?
  \item Now thing back to the traditional static equation -- what is
    the relative comparison period for this diff-in-diff? 
  \item We now consider the dyanamic versions of Equation
    \ref{eq:twfe}. Denote $D_{it}^{l} = 1(t - E_{i} = l)$
    \begin{equation}\label{eq:twfe_dyn}
      Y_{it} = \alpha_{i} + \lambda_{t} + \sum_{l \in -3,-2} D^{l}_{it}\beta_{l} +  \sum_{l \in 0,1,2,3} D^{l}_{it}\beta_{l} + \epsilon_{it}.
    \end{equation}
    Report the $\beta$ coefficients and their standard errors.
  \item Now, repeat this exercise, but consider the estimation
    group-by-group again. Focusing just on the Cohort Wave 8
    vs. Cohort Wave 11 comparison, how would you run the above
    specification? What coefficients are you able to estimate? Report
    these estimates. Now repeat and estimate $\beta_{0}$ for each of
    the groups. How do these estimates compare to your estimates from
    Equation \ref{eq:twfe_dyn}?
  \item Now focus on the estimate for $\beta_{-2}$ from Equation
    \ref{eq:twfe_dyn}. This is traditionally the pre-trend test. Sun
    and Abraham show that under the standard diff-in-diff assumptions,
    the $\beta_{-2}$ coefficient in Equation \ref{eq:twfe_dyn}
    specification, this coefficient is the weighted combination of
    multiple treatments in other periods. Denote $CATT_{e,l}$ as the
    average treatment effect $l$ periods from the initial treatment
    for the cohort of units first treated at time $e$. Then, Sun and Abraham show that
    \begin{equation}
      \beta_{-2} = \sum_{e=8}^{11}\omega_{e,-2}^{-2}CATT_{e,-2} + \sum_{l=-3,0,1,2,3}\sum_{e=8}^{11}\omega_{e,l}^{-2}CATT_{e,l} + \sum_{l'\in\{-4,-1\}}\sum_{e=8}^{11}\omega_{e,l'}^{-2}CATT_{e,l'}, 
    \end{equation}
    where the $\omega$ are weights that we can calculate. We can
    estimate these by replacing $Y_{it}$ in Equation \ref{eq:twfe_dyn}
    with $D_{i,t}^{l}1(E_{i}=e)$ as the outcome variable, and
    reporting the coefficient on $D^{-2}_{it}$. Do so for each $l$ and
    $e$. Your results should match Figure 2 in Sun and Abraham. How
    does this affect your interpretation of the pre-trend test?
  \item Finally, we estimate Sun and Abraham's alternative estimator,
    which avoids the contamination bias. This approach \emph{pools}
    our cohort-by-cohort comparison from before. First, we estimate
    \begin{equation}
      Y_{it} = \alpha_{i} + \lambda_{t} + \sum_{e = 8,9,10} \sum_{l =-3, l \not=-1}^{l=3} 1(E_{i} = e) \times D^{l}_{it}\delta_{e,l} + \epsilon_{it},
    \end{equation}
    where we exclude the last time period and treat the Cohort Wave 11
    as our control group. Take the $\delta_{e,l}$ estimates, and
    report $\delta_{e,0}$ for all 3 groups. The final estimate
    $\mu_{0}$ weights each of these $\delta$ by the cohort sample
    weight $\pi_{e} = Pr(E_{i} = e | l = 0)$. Report this estimate of
    $\mu_{0}$.
  \end{enumerate}
\end{enumerate}


\end{document}