\documentclass[11pt, a4paper]{article}
%\usepackage{geometry}
\usepackage[inner=1.5cm,outer=1.5cm,top=2.5cm,bottom=2.5cm]{geometry}
\pagestyle{empty}
\usepackage{graphicx}

\usepackage[usenames,dvipsnames]{color}
\definecolor{darkblue}{rgb}{0,0,.6}
\definecolor{darkred}{rgb}{.7,0,0}
\definecolor{darkgreen}{rgb}{0,.6,0}
\definecolor{red}{rgb}{.98,0,0}
\usepackage[colorlinks,pagebackref,pdfusetitle,urlcolor=darkblue,citecolor=darkblue,linkcolor=darkred,bookmarksnumbered,plainpages=false]{hyperref}
\renewcommand{\thefootnote}{\fnsymbol{footnote}}



\thispagestyle{plain}

%%%%%%%%%%%% LISTING %%%
\usepackage{listings}
\usepackage{caption}
\DeclareCaptionFont{white}{\color{white}}
\DeclareCaptionFormat{listing}{\colorbox{gray}{\parbox{\textwidth}{#1#2#3}}}
\captionsetup[lstlisting]{format=listing,labelfont=white,textfont=white}
\usepackage{verbatim} % used to display code
\usepackage{fancyvrb}
\usepackage{acronym}
\usepackage{amsthm}
\VerbatimFootnotes % Required, otherwise verbatim does not work in footnotes!



\definecolor{OliveGreen}{cmyk}{0.64,0,0.95,0.40}
\definecolor{CadetBlue}{cmyk}{0.62,0.57,0.23,0}
\definecolor{lightlightgray}{gray}{0.93}



\lstset{
%language=bash,                          % Code langugage
basicstyle=\ttfamily,                   % Code font, Examples: \footnotesize, \ttfamily
keywordstyle=\color{OliveGreen},        % Keywords font ('*' = uppercase)
commentstyle=\color{gray},              % Comments font
numbers=left,                           % Line nums position
numberstyle=\tiny,                      % Line-numbers fonts
stepnumber=1,                           % Step between two line-numbers
numbersep=5pt,                          % How far are line-numbers from code
backgroundcolor=\color{lightlightgray}, % Choose background color
frame=none,                             % A frame around the code
tabsize=2,                              % Default tab size
captionpos=t,                           % Caption-position = bottom
breaklines=true,                        % Automatic line breaking?
breakatwhitespace=false,                % Automatic breaks only at whitespace?
showspaces=false,                       % Dont make spaces visible
showtabs=false,                         % Dont make tabls visible
columns=flexible,                       % Column format
morekeywords={__global__, __device__},  % CUDA specific keywords
}

%%%%%%%%%%%%%%%%%%%%%%%%%%%%%%%%%%%%
\begin{document}
\begin{center}
  {\Large \textsc{Comprehensive Exam Practice Questions}}
  MGMT 737
\end{center}
\begin{center}
Spring 2023
\end{center}


In answering these questions, full marks are given for explanations,
not just right answers. Good luck!


\section{True/False/Uncertain}
For the following problems, respond True False or Uncertain, and
explain why. If you can refer to papers or results that is
particularly valuable.

\begin{itemize}
\item If something is randomly assigned conditional on variables $X$,
  controlling for those variables in a linear regression is sufficient
  to estimate the average treatment effect.
\item A difference-in-difference design with a single intervention
  time period suffers from negative weighting issues.
\item It is still possible to partially identify a treatment effect in
  a regression discontinuity setting if there is bunching in the
  running variable.
\item You should always cluster your standard errors in the most
  conservative way possible.
\item The parallel trends test is a good test for research design
  validity in the difference-in-difference setting.
\item If a treatment spills over and affects more than the treated
  units, it is not possible to estimate the average treatment effect
  of a treatment.
\item It is not possible to identify anything about the compliers in
  an instrumental variables regression.
\item The quantile regression estimation approach is robust to
  censoring in the outcome variable.
\item In a duration model, it is fine to throw out observations that
  are right censored.
\item A Bartik or shift-share research design rests on the validity of
  exogeneous shares.
\item A randomized experiment doesn't need to control for any
  confounding factors.
\item If you have an outcome variable $y$ with zeros and positive
  values, you can estimate a percentage effect by using $\log(1+y)$ as
  the outcome.
\item As long as your first stage F statistic is larger than 10, you
  should have no weak instrument problems.
\end{itemize}


\section{Application Questions}
\begin{enumerate}
\item In Angrist, Imbens and Rubin (1996), they study as an
  application the effect of military service on civilian
  mortality. The relevant variables are:
  \begin{itemize}
  \item $Z_{i}$: binary variable that person $i$ received a low draft lottery number (such
    that they more likely to be drafted)
  \item $D_{i}$: binary indicator that person $i$ served in the military
  \item $Y_{i}$: binary indicator that person $i$ died between 1974 and 1983 given lottery
  \end{itemize}
  \begin{enumerate}
    \item Write these variables in potential outcome notation. Describe, in words, what each potential outcome means. (Hint: there should be 4 total)
    \item Under what assumptions would a regression of $Y_{i}$ on $D_{i}$ yield the causal effect of military service on mortality rates? Write out this estimand using potential outcomes.
    \item Define the exclusion restriction for draft lottery numbers in terms of potential outcomes.
    \item List an example violation of this restriction.
    \item Under what assumptions would a regression of $Y_{I}$ on $Z_{i}$ yield the causal effect of draft lottery numbers on mortality rates? Write out this estimand using potential outcomes.
    \item Finally, under what additional assumptions could you use an
      instrumental variables approach with the draft lottery numbers
      to identify the effect of military service on mortality? Be
      precise in defining your assumptions in terms of potential
      outcomes.
    \item Imagine that the exclusion restriction is violated. Under
      what settings would this not cause significant bias in the IV
      estimator?
    \end{enumerate}
  \item Lee (2008) considers the impact of a Democrat winning on subsquent victory using a regression discontinuitiy design
  \begin{itemize}
  \item $Z_{i}$: running variable -- vote share margin of victory (RD at $Z_{i} = 0$)
  \item $D_{i}$: winning election
  \item $Y_{i}$: subseuquent victory in an election
    \begin{enumerate}
    \item Write out the estimand for the RD above
    \item Describe a way in which this design could be violated
    \item How would you estimate this effect? Describe the estimation
      procedure (not just what function you would use, but how it
      would be implemented. You do not need to be precise
      mathematically).
    \item A graduate student colleague of yours suggests running a
      linear regresion on both sides of the regression, using the full
      dataset, and then taking the predicted value at the cutoff for
      each regression. What issues might you have with that? Feel free
      to draw a picture.
    \item What issues arise with discrete running variables? How would
      you solve them?
    \end{enumerate}
  \end{itemize}
  \item Consider a random sample of individuals $i= 1, \ldots, n$,
    with treatment status $D_{i}$ and outcome $Y_{i}$.
    \begin{enumerate}
    \item Write out the individual treatment effects estimands using the potential outcome notation.
    \item Write the DAG for this effect
    \item Now imagine that $i= 1, \ldots, n$ instead indexes pairs of
      roommates in a college dorm, with $Y_{i} = (Y_{i1}, Y_{i2})$. If
      we thought treatments from roommates had spillover effects, how
      would you write the potential outcome? Define the different
      potential estimands you could construct using this notation
      (Hint: there should be 4).
    \item Write out the DAG for this setup.
    \item Would a regression of outcomes on the number of people in a
      room who are treated ($X_{i} = D_{i1} + D_{i2}$) capture any of
      these effects? Explain.
    \end{enumerate}
  \item Consider the effect of going to college $D_{i}$ on earnings
    $Y_{i}$.
    \begin{enumerate}
    \item We are given a number of covariates, $X_{i}$, and told that
      conditional on $X_{i}$, strict ignorability holds for $D_{i}$
      and the outcome. Write out what that means in potential outcome notation.
    \item Write down a DAG where this holds.
    \item How would you implement a p-score procedure to estimate the
      average treamtent effect of $D_{i}$ on $Y_{i}$, using the strict
      ignorability condition?
    \item You're now given data on occupation for these individuals
      ($W_{i}$). We might expect that occupation causes changes in
      earnings, and the choice of occupation is causally shifted by
      the decision to go to college. Add this variable to the DAG.
    \item What would happen to our causal estimate if we now added
      $W_{i}$ as a control to our estimation procedure?
    \end{enumerate}
  \item We consider the roll-out of a COVID-19 lockdown across some
    states, but not others. For $t \geq 0$, $D_{it} = 1$ for states in
    the treatment group, and $D_{it} = 0$ for the control states. For
    $t < 0$, $D_{it} = 0$ for everyone. We're interested in the effect
    on economic activity $Y_{i}$, and will use difference-in-differences.
    \begin{enumerate}
    \item Write out a simple parametric model that would let you
      identify the effect of this policy, while not necessarily
      assuming random assignment of $D_{i}$, given two time
      perods. What assumption is necessary?
    \item Write out the regression for how you would estimate a single
      treatment effect in the post-period for the treatment. (you may have done this in the last problem)
    \item What if $D_{it}$ had been randomly assigned? What could you do instead?
    \item Describe a test you can do to test the validity of this
      model. What do you need? What are some potential issues with
      this test?
    \item Now, you get access to policies that have been implemented
      at different times (a staggered roll-out). Describe in words how
      having a staggered roll-out provides a more robust
      identification approach.
    \item What issues might arise if you ran the same regression as in
      part b) above with the staggered roll-outs?  Describe in words,
      or with algebra, or with a graph (or all three). 
    \end{enumerate}


    \item You've been put in charge of evaluating the impact of Connecticut's policy to forgive medical debt. The policy targeted people whose household income is up to \$60,000.
    
    The state has tasked you with evaluating the impact of this policy on financial distress. You have been given a dataset with:
    \begin{itemize}
        \item $Y_{i}$: financial distress of individual $i$     
        \item $X_{i}$: an individual's household income
        \item $D_{i}$: the individual's initial amount of medical debt
    \end{itemize}
    
    \begin{enumerate}
        \item[a.] You have determined that you should estimate the effect of the policy on $Y$ using a regression discontinuity design. What assumptions are necessary for this design to be consistent? How would you estimate the effect of the policy using this design?
        \item[b.] Describe two tests you could do to test the assumptions necessary for the regression discontinuity design to be consistent.
        \item[c.] How could you use this regression discontinuity design to estimate the effect of changes in medical debt on financial distress?
        \item[d.] You now realize that the policy actually had two cutoffs: \$60k for a single person or \$120k for a married couple. Describe, either formally or in words, how you would adjust your estimation procedure to account for this. Do you need anymore assumptions?
        \item[e.] The state wants to know what would happen if they changed the cutoff for the policy to \$70k. What  (well-defended and supported) answer would you give them?
        \item[df.] You have now discovered a serious issue with the data: the data on household income is bucketed in bins of \$5000, so that everyone between 50,000 and 54,999 is reported as 50,000, everyone between 55,000 and 59,999 is reported as 55,000, etc. How would this affect your ability to estimate the effect of the policy using a regression discontinuity design? How could you address this issue?
    \end{enumerate}
    
   \item We want to estimate a linear regression model of the form:
    \begin{equation}
        Y_{i} = \beta_{0} + \beta_{1}X_{i} + \epsilon_{i},
    \end{equation}
    where $Y_{i}$ is the change in employment in the service sector in city $i$, $X_{i}$ is an indicator for whether the city passed a law that increased the minimum wage, and $\epsilon_{i}$ is an error term. Assume you observe $n$ cities. 
    
    \begin{enumerate}
        \item[a.] Assume that we could consistently estimate $\beta_{1}$ by using OLS. How would you calculate the standard errors for $\beta_{1}$ if you assumed that the errors, $\epsilon_{i}$, were homoskedastic?
        \item[b.] Now, we will consider a model where the effect of $X_{i}$ varies by city, but this heterogeneity is uncorrelated with whether the city passed a minimum wage law. Specifically, we assume that $\beta_{i} = \beta_{1} + (\beta_{i} - \beta_{1})$, $E(\beta_{i} - \beta_{1} | X_{i}) = 0$, and 
        \begin{equation}
            Y_{i} = \beta_{0} + \beta_{i}X_{i} + \epsilon_{i}.
        \end{equation}
        How would you estimate $\beta_{1}$ in this case? How would your standard errors change? If they change, please specify how you would estimate them.
        \item[c.] Now, we realize that the cities don't pass minimum wage laws on their own -- rather, the states passed the laws, making $X_{i}$ correlated within states. We assume we observe $S$ states and $n_{s}$ cities in each state. Observations in our regression are still at the city level.
        \begin{enumerate}
            \item[i] If we assume that there was no heterogeneity in $\beta_{1}$, how would we estimate standard errors for $\beta_{1}$? What assumptions are necessary?
            \item[ii] If we assume that there was heterogeneity in $\beta_{1}$ \emph{and it is correlated within cluster}, with $\beta_{i} - \beta_{1} = b_{s} + \tilde{\beta}_{i}$, how would we estimate standard errors for $\beta_{1}$? 
        \end{enumerate}
    \end{enumerate}
    
    
  \item You are interested in studying the impact of noncompete bans across states (A noncompete agreement is a contract between an employer and employee where the employee agrees not to work for a competitor for a specified period of time after leaving the company). You have a panel dataset with the following variables:
    \begin{itemize}
    \item $Y_{it}$: the wages of workers in state $i$ in year $t$
    \item $X_{it}$: an indicator for whether the state has an abortion ban in year $t$
    \end{itemize}
    You have $n$ states, and $T=3$ years of data. In the first year, no states have a ban. In the second year, $n_{e} = 20$ states have a ban put into place going forward, and in the third year, $n_{l} = 10$ more states have a ban, giving a total of 30 states with a ban, and 20 without.
    \begin{enumerate}
        \item[a.] Describe how you would estimate the effect of the ban on the average wage in the second year. What assumptions are necessary for this estimate to be consistent?
        \item[b.] How would you expand your approach from part a to estimate the effect of the abortion ban on the average wage using all time periods? What assumptions are necessary for this estimate to be consistent?
        \item[c.] Are there any tests you could do to test the assumptions necessary for your estimates to be consistent?
        \item[d.] One important consideration when studying state-level data is that individuals can cross into other states to work. How would this affect your estimates of the effect of the bans? Do you have any ideas on how you could quantify this effect?
        \item[e.] Now, you get another year of data, and in this year, all states have a ban. What assumptions would you need to make to estimate the effect of the ban on the average wage?
    \end{enumerate}    
\end{enumerate}
  
\end{document}